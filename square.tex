\documentclass[submission]{FPSAC2021}

\usepackage[utf8]{inputenc}
\usepackage[polish, english]{babel}
\usepackage[backend=bibtex, url = false, doi=false]{biblatex}
\usepackage{tikz}
\usetikzlibrary{calc}
\usepackage{pgfplots}
\usepackage[labelfont={normalsize}]{caption,subfig}
\usepackage{amsmath,amssymb, amsthm}
\usepackage{float}
\usepackage[backend=bibtex]{biblatex}
\usepackage[backgroundcolor=yellow]{todonotes}
\usepackage{cleveref}
\usepackage[normalem]{ulem}

\bibliography{square.bib}

\newtheorem{thm}{Theorem}
\newtheorem{theorem}{Theorem}
\newtheorem*{conjecture}{Goulden--Rattan conjecture}
\newtheorem{con}{Conjecture}

\DeclareMathOperator{\degg}{deg}
\DeclareMathOperator{\odd}{odd}
\DeclareMathOperator{\even}{even}
\DeclareMathOperator{\rest}{rest}

\newcommand{\nast}
{
	\sigma
}

\newcommand{\lewodd}[4]
{
	\draw[color=#4] [densely dashed] (#1) to [bend left=#3] (#2);
}

\newcommand{\prawolewodd}[4]
{
	\lewodd{#1}{$.5*(#1)+.5*(#2)$}{#3}{#4}
	\lewodd{#2}{$.5*(#1)+.5*(#2)$}{#3}{#4}
}

\newcommand{\torus}[2]
{

	\centering
	\subfloat[]
	{
		\begin{tikzpicture}[scale=#1]
		\coordinate (x) at (-5,-1);
		\coordinate (y) at (-2,-2);
		\coordinate (z) at (2,-2);
		\coordinate (t) at (5,-1);
		\coordinate (p) at (-5,2);
		\coordinate (q) at (0,3);
		\coordinate (r) at (5,2);
		\coordinate (yz) at (-.7,-.85);
		\coordinate (zy) at (1.2,-3.9);
		\coordinate (gora) at (0,.2);
		\coordinate (dol) at (0,-1.1);
		\draw[color=black, ultra thick] (0,0) ellipse (8 and 4);
		\draw[color=black, ultra thick] (5,1) to [bend left=40] (-5,1);
		\draw[color=black, ultra thick] (3.5,0) to [bend right=40] (-3.5,0);
		\prosto{x}{y}{black}
		\prosto{y}{z}{black}
		\prosto{z}{t}{black}
		\lewo{y}{yz}{30}{blue}
		\lewo{z}{zy}{40}{blue}
		\prawolewodd{zy}{yz}{40}{blue}
		\lewo{x}{p}{60}{black}
		\lewo{p}{r}{25}{black}
		\lewo{r}{t}{60}{black}
		\wierzcholek{x}{black}{b_2}{gora}
		\wierzcholek{y}{white}{w_1}{dol}
		\wierzcholek{z}{black}{b_1}{gora}
		\wierzcholek{t}{white}{w_2}{dol}
		\end{tikzpicture}
		\label{fig:rystorus}
	}
	\hfill
	\subfloat[]
	{
		\begin{tikzpicture}[scale=#2]
			\punkty		
			\prosto{a}{b}{black}
			\prosto{b}{c}{black}
			\prosto{c}{d}{black}
			\prosto{d}{a}{black}
			\lewoprawo{c}{d}{70}{blue}
			\wierzcholek{a}{black}{b_2}{gora}
			\wierzcholek{c}{black}{b_1}{dol}
			\wierzcholek{b}{white}{w_2}{gora}
			\wierzcholek{d}{white}{w_1}{gora}
		\end{tikzpicture}
		\label{fig:rysnormal}
	}	
}

\newcommand{\punkty}
{
	\coordinate (a) at (0,2);
	\coordinate (b) at (1.4,0);
	\coordinate (zab) at (2,0);
	\coordinate (c) at (0,-2);
	\coordinate (d) at (-1.4,0);
	\coordinate (zad) at (-2,0);
	\coordinate (gora) at (0,.2);
	\coordinate (dol) at (0,-.9);
}

\newcommand{\prosto}[3]
{
	\draw[color=#3] (#1) to [bend left=0] (#2);
}

\newcommand{\lewo}[4]
{
	\draw[color=#4] (#1) to [bend left=#3] (#2);
}

\newcommand{\prawo}[4]
{
	\draw[color=#4] (#1) to [bend right=#3] (#2);
}

\newcommand{\prawolewo}[4]
{
	\prawo{#1}{$.5*(#1)+.5*(#2)$}{#3}{#4}
	\lewo{$.5*(#1)+.5*(#2)$}{#2}{#3}{#4}
}

\newcommand{\lewoprawo}[4]
{
	\lewo{#1}{$.5*(#1)+.5*(#2)$}{#3}{#4}
	\prawo{$.5*(#1)+.5*(#2)$}{#2}{#3}{#4}
}

\newcommand{\szeroko}[4]
{
	\draw[color=#4] (#1) to [bend left=45] (#3);
	\draw[color=#4] (#2) to [bend right=45] (#3);
}

\newcommand{\waskoszeroko}[4]
{
	\draw[color=#4] (#1) to [bend left=0] (#3);
	\draw[color=#4] (#2) to [bend right=45] (#3);
}

\newcommand{\waskoszerokodd}[4]
{
	\draw[color=#4] [densely dashed] (#1) to [bend left=0] (#3);
	\draw[color=#4] [densely dashed] (#2) to [bend right=45] (#3);
}


\newcommand{\prostodd}[3]
{
	\draw[color=#3] (#1) [densely dashed] to [bend left=0] (#2);
}

\newcommand{\bialy}[1]
{
	\draw[black,fill=white] (#1) circle (5pt);
}

\newcommand{\wierzcholek}[4]
{
\draw[black,fill=#2] (#1) circle (5pt) node[above] at ($(#1)+(#4)$) {\tiny \large $#3$};
}

\newcommand{\rysodd}[1]
{	
	\centering
	\subfloat[]
	{
		\begin{tikzpicture}[scale=#1]
			\punkty		
			\prosto{a}{b}{black}
			\prosto{b}{c}{red}
			\prosto{c}{d}{red}
			\prosto{d}{a}{black}
			\prawolewo{c}{d}{45}{red}
			\wierzcholek{a}{black}{b_2}{gora}
			\wierzcholek{c}{black}{b_1}{dol}
			\wierzcholek{b}{white}{w_2}{gora}
			\wierzcholek{d}{white}{w_1}{gora}
		\end{tikzpicture}
		\label{fig:rysodda}
	}
	\hfill
	\subfloat[]
	{
		\begin{tikzpicture}[scale=#1]
			\punkty		
			\prosto{a}{b}{black}
			\szeroko{$.5*(a)+.5*(b)$}{c}{$(zab)+(0.0,-0.2)$}{red}
			\prosto{c}{$.5*(a)+.5*(d)$}{red}
			\prosto{d}{a}{black}
			\prawolewo{c}{$.5*(a)+.5*(d)$}{35}{red}
			\wierzcholek{a}{black}{b_2}{gora}
			\wierzcholek{c}{black}{b_1}{dol}
			\wierzcholek{b}{white}{w_2}{gora}
			\wierzcholek{d}{white}{w_1}{gora}
		\end{tikzpicture}
		\label{fig:rysoddb}
	}
	\hfill
	\subfloat[]
	{
		\begin{tikzpicture}[scale=#1]
			\punkty		
			\prosto{a}{b}{black}
			\szeroko{a}{c}{zab}{red}
			\prosto{c}{a}{red}
			\prosto{d}{a}{black}
			\prawolewo{c}{a}{25}{red}
			\wierzcholek{a}{black}{b_2}{gora}
			\wierzcholek{c}{black}{b_1}{dol}
			\wierzcholek{b}{white}{w_2}{gora}
			\wierzcholek{d}{white}{w_1}{gora}
		\end{tikzpicture}
		\label{fig:rysoddc}
	}
	\hfill
	\subfloat[]
	{
		\begin{tikzpicture}[scale=#1]
			\punkty		
			\prosto{a}{b}{black}
			\szeroko{a}{c}{zab}{red}
			\prosto{c}{a}{red}
			\prosto{d}{a}{black}
			\prawolewo{c}{a}{25}{red}
			\wierzcholek{a}{black}{b}{gora}
			\wierzcholek{c}{white}{w_3}{dol}
			\wierzcholek{b}{white}{w_2}{gora}
			\wierzcholek{d}{white}{w_1}{gora}
		\end{tikzpicture}
		\label{fig:rysoddd}
	}
}

\newcommand{\ryseven}[1]
{	
	\centering
	\subfloat[]
	{
		\begin{tikzpicture}[scale=#1]
			\punkty		
			\prosto{a}{b}{red}
			\prosto{b}{c}{black}
			\prosto{c}{d}{black}
			\prosto{d}{a}{red}
			\prawolewo{c}{d}{55}{black}
			\wierzcholek{a}{black}{b_2}{gora}
			\wierzcholek{c}{black}{b_1}{dol}
			\wierzcholek{b}{white}{w_2}{gora}
			\wierzcholek{d}{white}{w_1}{gora}
		\end{tikzpicture}
		\label{fig:rysevena}
	}
	\hfill
	\subfloat[]
	{
		\begin{tikzpicture}[scale=#1]
			\punkty		
			\prosto{a}{$.5*(b)+.5*(c)$}{red}
			\prosto{b}{c}{black}
			\prosto{c}{d}{black}
			\szeroko{$.5*(c)+.5*(d)+(0.4,-0.0)$}{a}{$(zad)+(0.0,0.2)$}{red}
			\prawo{c}{d}{30}{black}
			\wierzcholek{a}{black}{b_2}{gora}
			\wierzcholek{c}{black}{b_1}{dol}
			\wierzcholek{b}{white}{w_2}{gora}
			\wierzcholek{d}{white}{w_1}{gora}
		\end{tikzpicture}
		\label{fig:rysevenb}
	}
	\hfill
	\subfloat[]
	{
		\begin{tikzpicture}[scale=#1]
			\punkty		
			\prosto{b}{c}{black}
			\lewo{c}{d}{35}{black}
			\lewo{a}{c}{20}{red}
			\prawo{a}{c}{40}{red}
			\prawo{c}{d}{35}{black}
			\wierzcholek{a}{black}{b_2}{gora}
			\wierzcholek{c}{black}{b_1}{dol}
			\wierzcholek{b}{white}{w_2}{gora}
			\wierzcholek{d}{white}{w_1}{gora}
		\end{tikzpicture}
		\label{fig:rysevenc}
	}
	\hfill
	\subfloat[]
	{
		\begin{tikzpicture}[scale=#1]
			\punkty		
			\prosto{b}{c}{black}
			\lewo{c}{d}{35}{black}
			\lewo{a}{c}{20}{red}
			\prawo{a}{c}{40}{red}
			\prawo{c}{d}{35}{black}
			\wierzcholek{a}{white}{w_3}{gora}
			\wierzcholek{c}{black}{b}{dol}
			\wierzcholek{b}{white}{w_2}{gora}
			\wierzcholek{d}{white}{w_1}{gora}
		\end{tikzpicture}
		\label{fig:rysevend}
	}
}

\newcommand{\rysrest}[1]
{	
	\centering
	\subfloat[]
	{
		\begin{tikzpicture}[scale=#1]
			\punkty		
			\prosto{a}{b}{black}
			\prostodd{b}{c}{blue}
			\prosto{c}{d}{red}
			\prosto{d}{a}{black}
			\prawolewo{c}{d}{55}{red}
			\wierzcholek{a}{black}{b_2}{gora}
			\wierzcholek{c}{black}{b_1}{dol}
			\wierzcholek{b}{white}{w_2}{gora}
			\wierzcholek{d}{white}{w_1}{gora}
		\end{tikzpicture}
		\label{fig:rysresta}
	}
	\hfill
	\subfloat[]
	{
		\begin{tikzpicture}[scale=#1]
			\punkty		
			\prosto{a}{b}{black}
			\waskoszerokodd{$.5*(a)+.5*(b)$}{c}{$.6*(b)+.4*(c)$}{blue}
			\prosto{c}{$.5*(a)+.5*(d)$}{red}
			\prosto{d}{a}{black}
			\prawolewo{c}{$.5*(a)+.5*(d)$}{35}{red}
			\wierzcholek{a}{black}{b_2}{gora}
			\wierzcholek{c}{black}{b_1}{dol}
			\wierzcholek{b}{white}{w_2}{gora}
			\wierzcholek{d}{white}{w_1}{gora}
		\end{tikzpicture}
		\label{fig:rysrestb}
	}
	\hfill
	\subfloat[]
	{
		\begin{tikzpicture}[scale=#1]
			\punkty		
			\prosto{a}{b}{black}
			\waskoszerokodd{a}{c}{$.6*(b)+.4*(c)$}{blue}
			\prosto{c}{a}{red}
			\prosto{d}{a}{black}
			\prawolewo{c}{a}{25}{red}
			\wierzcholek{a}{black}{b_2}{gora}
			\wierzcholek{c}{black}{b_1}{dol}
			\wierzcholek{b}{white}{w_2}{gora}
			\wierzcholek{d}{white}{w_1}{gora}
		\end{tikzpicture}
		\label{fig:rysrestc}
	}
	\hfill
	\subfloat[]
	{
		\begin{tikzpicture}[scale=#1]
			\punkty		
			\prosto{a}{b}{black}
			\waskoszerokodd{a}{c}{$.6*(b)+.4*(c)$}{blue}
			\prosto{c}{a}{red}
			\prosto{d}{a}{black}
			\prawolewo{c}{a}{25}{red}
			\wierzcholek{a}{black}{b}{gora}
			\wierzcholek{c}{white}{w_3}{dol}
			\wierzcholek{b}{white}{w_2}{gora}
			\wierzcholek{d}{white}{w_1}{gora}
		\end{tikzpicture}
		\label{fig:rysrestd}
	}
}

\newcommand{\sliding}[1]
{
	\centering
	\subfloat[]
	{
		\begin{tikzpicture}[scale=#1]
			\punkty		
			\prosto{d}{a}{red}
			\lewoprawo{d}{a}{40}{red}
			\lewo{a}{d}{65}{red}
			\lewoprawo{b}{d}{35}{black}
			\prawo{d}{$(0,0)$}{35}{black}
			\prosto{$(0,0)$}{$(1,0.6)$}{black}
			\lewo{$(1,0.6)$}{b}{90}{black}
			\lewo{b}{d}{65}{black}
			\wierzcholek{a}{black}{}{gora}
			\wierzcholek{b}{black}{}{gora}
			\wierzcholek{d}{black}{}{$(dol)+(-0.2,0.0)$}
		\end{tikzpicture}
		\label{fig:rysprzed}
	}
	\subfloat[]
	{
		\begin{tikzpicture}[scale=#1]
			\punkty		
			\prosto{$(d)+(0.7,0.23)$}{a}{red}
			\lewoprawo{$(d)+(0.7,0.23)$}{a}{30}{red}
			\lewo{a}{$(d)+(0.7,-0.23)$}{45}{red}
			\lewoprawo{b}{d}{35}{black}
			\prawo{d}{$(0,0)$}{35}{black}
			\prosto{$(0,0)$}{$(1,0.6)$}{black}
			\lewo{$(1,0.6)$}{b}{90}{black}
			\lewo{b}{d}{65}{black}
			\wierzcholek{a}{black}{}{gora}
			\wierzcholek{b}{black}{}{gora}
			\wierzcholek{d}{black}{}{$(dol)+(-0.2,0.0)$}
		\end{tikzpicture}
		\label{fig:rysw}
	}
	\subfloat[]
	{
		\begin{tikzpicture}[scale=#1]
			\punkty		
			\prosto{b}{a}{red}
			\lewoprawo{b}{a}{40}{red}
			\lewo{a}{b}{100}{red}
			\lewoprawo{b}{d}{35}{black}
			\prawo{d}{$(0,0)$}{35}{black}
			\prosto{$(0,0)$}{$(1,0.6)$}{black}
			\lewo{$(1,0.6)$}{b}{90}{black}
			\lewo{b}{d}{65}{black}
			\wierzcholek{a}{black}{}{gora}
			\wierzcholek{b}{black}{}{gora}
			\wierzcholek{d}{black}{}{$(dol)+(-0.2,0.0)$}
		\end{tikzpicture}
		\label{fig:ryspo}
	}
	\subfloat[]
	{
		\begin{tikzpicture}[scale=#1]
			\punkty		
			\lewoprawo{b}{d}{35}{black}
			\prawo{d}{$(0,0)$}{35}{black}
			\prosto{$(0,0)$}{$(1,0.6)$}{black}
			\lewo{$(1,0.6)$}{b}{90}{black}
			\lewo{b}{d}{65}{black}
			\wierzcholek{a}{black}{}{gora}
			\wierzcholek{b}{black}{}{gora}
			\wierzcholek{d}{black}{}{gora}
			\draw[red,fill=black] (d) circle (0pt) node[above] at ($(d)+(-0.4,-0.3)$) {\tiny \small $1$};
			\draw[red,fill=black] (d) circle (0pt) node[above] at ($(d)+(0.5,-0.3)$) {\tiny \small $5$};
			\draw[red,fill=black] (d) circle (0pt) node[above] at ($(d)+(0.75,-0.73)$) {\tiny \small $3$};
			\draw[red,fill=black] (b) circle (0pt) node[above] at ($(b)+(0.4,-0.3)$) {\tiny \small $6$};
			\draw[red,fill=black] (b) circle (0pt) node[above] at ($(b)+(-0.35,-0.1)$) {\tiny \small $2$};
			\draw[red,fill=black] (b) circle (0pt) node[above] at ($(b)+(-0.75,-0.72)$) {\tiny \small $4$};
			\draw[red,fill=black] (a) circle (0pt) node[above] at ($(a)+(-0.50,-0.45)$) {\tiny \small $1'$};
		\end{tikzpicture}
		\label{fig:rysnum}
	}
}

\title{Quadratic coefficients of Goulden--Rattan character polynomials}

\author[Mikołaj Marciniak]{Mikołaj Marciniak\thanks{\href{mailto:marciniak@mat.umk.pl}{marciniak@mat.umk.pl}. Mikołaj Marciniak was partially supported by Narodowe Centrum Nauki, grant number 2017/26/A/ST1/00189.}\addressmark{1}}

\address{\addressmark{1}Interdisciplinary Doctoral School “Academia Copernicana”, Faculty of Mathematics and Computer Science, Nicolaus Copernicus University in Toruń, ul.~Chopina 12/18, 87-100 Toruń, Poland}

\received{\today}

\abstract{ Goulden--Rattan polynomials give the exact value of the subdominant
    part of the normalized characters of the symmetric groups in terms certain
    quantities $(C_i)$ which describe the macroscopic shape of the Young diagram.
    Goulden--Rattan positivity conjecture states that the coefficients of these
    polynomials are positive rational numbers with small denominators. We prove a
    special case of this conjecture for the coefficient of the quadratic term
    $C_2^2$ by applying certain bijections involving maps (i.e., graphs drawn on
    surfaces).}

\keywords{characters of the symmetric groups, free cumulants, Kerov polynomials, Goulden--Rattan polynomials, maps}

\begin{document}
\maketitle
\section{Introduction}

A full version of this extended abstract will be available soon 
as a separate paper \cite{Mar21}.

\subsection{Normalized characters}

If $k\leq n$ are natural numbers, then any permutation $\pi \in S_k$ can also be
treated as an element of the larger symmetric group $S_n$ by adding $n-k$ additional
fixpoints. For any permutation $\pi \in S_k$ and any irreducible representation
$\rho^{\lambda}$ of the symmetric group $S_n$ which corresponds to the
Young diagram $\lambda$, we define \emph{the normalized character}
$$
\Sigma_{\pi}(\lambda)
= \begin{cases} 
          n(n-1)\cdots(n-k+1)\frac{\operatorname{Tr}\rho^{\lambda}(\pi)}{\text{dimension of }\rho^{\lambda}} & \text{for } k \leq n,\\
          0 & \text{otherwise.}
\end{cases}
$$
Particularly interesting are the character values on the cycles, therefore we
will use the shorthand notation
$$\Sigma_{k}(\lambda)=\Sigma_{(1,2,\ldots,k)}(\lambda).$$

\subsection{Free cumulants}

\emph{The free cumulants} are an important tool of the free probability theory
\cite{VDN92} and the random matrix theory \cite{Voi91}. 
In the context of the representation theory of the symmetric groups they can be defined as follows, see \cite{Bia03}. For a Young diagram $\lambda$ we define its free cumulants $R_2(\lambda), R_3(\lambda), \ldots$ as
$$R_k(\lambda)=\lim_{s\to\infty} \frac{1}{s^k}\Sigma_{k-1}(s\lambda),$$
where the diagram $s\lambda$ is created from the diagram $\lambda$ by 
dividing each box of $\lambda$ into a square $s\times s$. 

The free cumulants are very helpful for studying asymptotic behaviour of the characters on a cycle with the length $k$ when the size of the Young diagram tends to infinity \cite{Bia98}.

\subsection{Kerov character polynomials}

Kerov formulated the following result: for each permutation $\pi$ and any Young
diagram $\lambda$, the normalized character $\Sigma_{\pi}(\lambda)$ is equal to
the value of some polynomial $K_{\pi}(R_2(\lambda), R_3(\lambda), \ldots)$
(called now \emph{Kerov character polynomial}) with integer coefficients. The
first published proof of this fact was provided by Biane \cite{Bia03}. The
Kerov character polynomial is \emph{universal} because it does not depend on the
choice of $\lambda$. We investigate the values of the characters on the cycles, therefore 
for $\pi=(1, 2, \ldots, k)$ we use special simplified notation
\begin{align}
\label{kerpol}
\Sigma_k=K_k(R_2, R_3, \ldots)
\end{align}
for such Kerov polynomials. The first few Kerov polynomials $K_k$ are as follows:
\begin{align*}
\Sigma_1&=R_2,\\
\Sigma_2&=R_3,\\
\Sigma_3&=R_4+R_2,\\
\Sigma_4&=R_5+5R_3,\\
\Sigma_5&=R_6+15R_4+5R_2^2+8R_2,\\
\Sigma_6&=R_7+35R_5+35R_3R_2+84R_3, \\
\Sigma_7&=R_8+180R_2+224R_2^2+14R_2^3+56R_3^2+469R_4+84R_2R_4+70R_6.
%\Sigma_8&=3044R_3+2688R_2R_3+126R_2^2R_3+252R_3R_4+1869R_5+168R_2R_5+126R_7+R_9.
\end{align*}

Kerov conjectured that the coefficients of the polynomial $K_k$ are non-negative integers.
Goulden and Rattan \cite{GR05} found an explicit formula for the coefficients of the Kerov
polynomial $K_k$; unfortunately, their formula was complicated and did not give
any combinatorial interpretation to the coefficients. Later, F\'{e}ray
proved positivity \cite{Fer09} and together with Dołęga and Śniady found the
combinatorial interpretation of the coefficients \cite{DFS10}.

\subsection{Goulden--Rattan conjecture}

Goulden and Rattan \cite{GR05} introduced a family of functions 
$C_2,C_3,\dots$ on the set of Young diagrams given by
\begin{align}
\label{cformula}
C_k=\sum_{\substack{j_2,j_3,\ldots \geq 0 \\ 2j_2+3j_3+\cdots=k}}
(j_2+j_3+\cdots)!\prod_{i\geq 2} \frac{\big( (i-1)R_i\big)^{j_i}}{j_i!}
\end{align}
for $k\geq 2$.
The aforementioned formula of Goulden and Rattan for the Kerov polynomials 
was naturally expressed in terms of these 
quantities $C_2,C_3,\dots$ \cite{GR05}. 
More specifically, they constructed an explicit polynomial $L_k$ with
rational coefficients such that 
\begin{align}
\label{grpol}
\Sigma_k-R_{k+1}=L_k(C_2, C_3, \ldots).
\end{align}
These polynomials are called \emph{the Goulden--Rattan polynomials}. They formulated the following conjecture:
\begin{conjecture}
\label{hipotezaGR}
The coefficients of the Goulden--Rattan polynomials are non-negative numbers. 
\end{conjecture}
The first few Goulden--Rattan polynomials are as follows \cite{GR05}:
\begin{align*}
\Sigma_1&-R_2=0\\
\Sigma_2&-R_3=0\\
\Sigma_3&-R_4=C_2,\\
\Sigma_4&-R_5=\frac{5}{2}C_3,\\
\Sigma_5&-R_6=5C_4+8C_2,\\
\Sigma_6&-R_7=\frac{35}{4}C_5+42C_3,\\
\Sigma_7&-R_8=14C_6+\frac{469}{3}C_4+\frac{203}{3}C_2^2+180C_2.
%\Sigma_8&-R_9=21C_6+\frac{1869}{4}C_5+\frac{819}{2}C_3C_2+1522C_2.
\end{align*}

Linear coefficients of the Goulden--Rattan polynomial are non-negative, because
they are equal to certain scaled coefficients of the Kerov polynomial:
\[ [ C_j ] L_k=(j-1) [ R_j ] K_k. \]
In this
paper we will prove that the coefficient of $C_2^2$ is non-negative.
Using the same methods we will prove in a separate paper \cite{Mar21} that 
any square coefficients $[ C_i C_j ] L_k$ are non-negative. 
The next step towards the proof of Goulden--Rattan conjecture would be understand 
the cubic coefficients $ [ C_i C_j C_u ] L_k$; we hope that 
our methods will still be applicable there, nevertheless there seem to be 
some difficulties related to the inclusion-exclusion principle.

\section{Coefficients of the Goulden--Rattan polynomial}

\subsection{Maps}

By \emph{a map} we will understand a bipartite graph drawn without intersections
on an oriented and connected surface with minimal genus. The maps which we
consider are \emph{bicolored}, i.e., each vertex is coloured black or white, with
the edges connecting the vertices of the opposite colors. 
An example of a map is shown in \cref{fig:mapa}.

\begin{figure}
\torus{0.65}{1.0}
\caption
{
	\protect\subref{fig:rystorus} An example of a map with $4$ vertices and $5$ edges drawn on a torus.  \\
	\protect\subref{fig:rysnormal} The same map drawn for simplicity on the plane. 
}
\label{fig:mapa}
\end{figure}

\emph{An expander} \cite[Appendix A.1]{Sni19} 
 is a map with the following properties.
\begin{itemize}

\item It has one face and one edge (\emph{the root}) is decorated.

\item Each black vertex has assigned a \emph{weight} which is a natural number.
We assume that each non-empty proper subset of the set of black vertices has
more white neighbours than the sum of its weights.

\item The sum of all weights is equal to the number of white vertices.

\end{itemize}
The map from \cref{fig:mapa} is also the expander if each black vertex has weight $1$ (the root is not marked).

Using the Euler characteristic we get
\begin{align}
\label{euler}
2-2g=\chi=V-k+1
\end{align}
where $g$ denotes the genus of the surface and $V$ denotes the number of the vertices.

Two following two theorems give a combinatorial interpretation to the linear
and square coefficients of the Kerov character polynomials.

\begin{theorem}
\label{thm2}
For all integers $l\geq2$ and $k\geq 1$ the coefficient $[R_l] K_k$ is equal to the number of expanders with $k$ edges, $l-1$ white vertices and one black vertex with the weight $l-1$.
\end{theorem}

\begin{theorem}
\label{thm4} 

For all integers $l_1, l_2\geq 2$ and $k\geq 1$ the coefficient
$[R_{l_1} R_{l_2}] K_k$ is equal to the number of expanders with $k$ edges,
$l_1+l_2-2$ white vertices and two black vertices with weights $l_1-1$ and
$l_2-1$.
\end{theorem}

\subsection{Relationship between coefficients of Goulden--Rattan polynomials 
and coefficients of Kerov polynomials}

The formula \eqref{cformula} allows us to express $(C_i)$ in terms of free cumulants;
we see that the coefficients of the terms $R_i R_j$ and $R_{i+j}$ in the expressions
$C_{i} C_{j}$ and $C_{i+j}$ are given by
\begin{align*}
C_i C_j &= (i-1)(j-1) R_i R_j + 0  R_{i+j} + \text{(sum of other terms)},\\
C_{i+j} &=2(i-1)(j-1)R_iR_j+(i+j-1)R_{i+j}+\text{(sum of other terms)} 
\qquad \text{for $i\neq j$.}
\end{align*}
Moreover, if we consider the terms $C_{i_1} C_{i_2}\cdots C_{i_t}$, then 
only the terms $C_i C_j$ and $C_{i+j}$ contain the expression $R_i R_j$ or $R_{i+j}$.
It follows that the quadratic coefficients of the Goulden--Rattan polynomial are 
related to the coefficients of the Kerov polynomial via
\begin{align*}
\frac{\partial^2L_k}{\partial{C_i}\partial{C_j}}\Bigg|_{0=C_1=C_2=\cdots}
&=\frac{1}{(i-1)(j-1)}\frac{\partial^2K_k}{\partial{R_i}\partial{R_j}}\Bigg|_{0=R_1=R_2=\cdots}
-2\frac{\partial L_k}{\partial{C_{i+j}}}\Bigg|_{0=C_1=C_2=\cdots}\\
&=\frac{1}{(i-1)(j-1)}\frac{\partial^2K_k}{\partial{R_i}\partial{R_j}}\Bigg|_{0=R_1=R_2=\cdots}
-\frac{2}{(i+j-1)}\frac{\partial K_k}{\partial{R_{i+j}}}\Bigg|_{0=R_1=R_2=\cdots};
\end{align*}
note that the the above equality is valid also in the case $i=j$. Thus
\begin{align}
\label{formula}
[C_j^2]L_k &=\frac{1}{(j-1)^2}[R_j^2]K_k-\frac{1}{2j-1}[R_{2j}]K_k \\
\intertext{and}
[C_iC_j]L_k &=\frac{1}{(i-1)(j-1)}[R_iR_j]K_k-\frac{2}{i+j-1}[R_{i+j}]K_k
\qquad \text{for $i\neq j$.}
\end{align}

\pagebreak[1] 

\section{The main result} 

Let $Y_k(u)$ denote the set of expanders with $k$
edges, $u-1$ white vertices and one black vertex. 
Let $X_k(i, j)$ denote the set of expanders with $k$ edges, $i+j-2$ white
vertices and two black vertices with weights $i-1$ and $j-1$. Using  \cref{thm2}
and \cref{thm4} we can also reformulate the Goulden--Rattan conjecture
for the square coefficients in the expander language, as follows.

\begin{con} 
    \label{con:GJ2}
Let $i, j$ be natural numbers. Then 
$$(2j-1)\ \left\| X_k(j, j) \right\| \geq (j-1)^2\ \left\| Y_k(2j) \right\|$$
and
$$(i+j-1)\ \left\| X_k(i, j) \right\| \geq 2(i-1)(j-1)\ \left\| Y_k(i+j) \right\|$$
for any natural number $k$. 
\end{con}
These inequalities are equivalent to the positivity 
of the coefficients $[C_j^2] L_k$ and $[C_i C_j] L_k$. 
In this text we prove the positivity in the special case $i=j=2$. 
We postpone the proof of \cref{con:GJ2} in its general form to the forthcoming full version of this paper.

Using \cref{formula} 
we can calculate several examples of the coefficient at $C_2^2$ of 
the Goulden--Rattan polynomials
\begin{align*}
\begin{split}
[C_2^2] L_4 &=0-0=0,\\
[C_2^2] L_6 &=0-0=0, \\
[C_2^2] L_8 &=0-0=0,
\end{split}
\begin{split}
[C_2^2] L_5 &=5-\frac{1}{3}15=0,\\
[C_2^2] L_7 &=224-\frac{1}{3}469=\frac{203}{3}.\\
\end{split}
\end{align*}
Note that if $k$ is even then $[C_2^2] L_k=0$ because there does not exists an expander with $4$ vertices and an odd number of edges, since
$2-2g=2j-k+1$ by \cref{euler}. Thus we can assume that the number of edges $k$ is odd. 

Let 
\begin{align}
\label{xdef}
X_k &= X_k(2, 2), \\
\label{ydef}
Y_k &= Y_k(4).
\end{align}
The set $X_k$ consists of expanders with $2$ black vertices and $2$ white vertices
such that each black vertex is connected with the both white vertices;
each black vertex necessarily has weight equal to $1$.
The set $Y_k$ consists of expanders with one black vertex (which necessarily has the weight $3$) connected
with all $3$ white vertices. From this moment the weights of the black vertices
will not be used anymore.

In order to prove \cref{con:GJ2} in the special case $i=j=2$ we will show that 
$$3\big\|X_k\big\| \geq \big\|Y_k\big\|.$$

\subsection{Edge sliding}

We define \emph{the edge sliding} on a graph as follows. 
We start from a graph $G$ without loops drawn on a surface 
with selected set of special edges. 
We assume that each special edge has selected its one end 
and its direction (left or right).

By the rest of the graph $G'$ we will understand the graph 
with the special edges removed. A corner of
$G'$ is an inner angle in a vertex between 
two neighboring edges of $G'$. 
We denote by $\sigma$ the permutation 
on the set of corners of $G'$ such that 
each cycle of $\sigma$ corresponds to the
corners which belong to some face of $G'$, 
arranged in the cyclic order (i.e., by going along
the boundary of the face of $G'$, touching it with the right-hand side).
For the example on \cref{fig:rysnum} we have 
$\sigma = ( 1' )( 1, 2, 3, 4, 5, 6 ).$


\begin{figure}
\sliding{0.95}
\caption
{
	An example of the edge sliding.
	\protect\subref{fig:rysprzed} A graph with special edges coloured red.
	\protect\subref{fig:rysw} The graph during the edge sliding. 
	Each special edge has the selected end at the bottom and 
	the direction right. 
	\protect\subref{fig:ryspo} The graph after the edge sliding.
	\protect\subref{fig:rysnum} The graph without the special edges with
	numbered order of the corners.
}
\label{fig:sliding}
\end{figure}

Each corner $c$ of a graph $G'$ can contain some selected ends 
of special edges. The output of the edge sliding is defined 
as the graph $G$ in which these selected ends of special edges 
in a corner $c$ are slided to the next corner $\sigma(c)$ 
if the direction is right and previous corner $\sigma^{-1}(c)$ 
if the direction is left. (see \cref{fig:sliding} for an example).

A collision of selected edge end $e$ is possible in two cases. First case, when exists the end of a special edge on the relevant side in the same corner, which does not slide in the same direction. Second case, when in the corner to which move the end $e$, exists the end of a special edge which slide in the opposite direction. 

Additionally, we also assume that special edges, their ends and 
their directions are selected in such a special way that 
during sliding the edges will have not collision and 
the graph obtained at the end will not contain a loop. 

The edge sliding is a bijection. The inversion of the edge sliding is 
the edge sliding with the selection for the same special edges, 
the same ends of these edges and the opposite directions of these edges.
In addition, the edge sliding on a graph does not change the
number of faces of this graph.  

%\subsection{Edge sliding}
%
%\todo[inline]{szybkie pytanie: czy krawędzi łączące wierzchołek z samym sobą są ok?}
%
%\todo{Pewnie bardziej zgodne z tradycją językową byłoby nazwanie tego 
%\emph{clockwise edge sliding} albo \emph{counterclockwise edge sliding}}
%
%
%We define \emph{the right-handed edge sliding} on a graph as follows. We %start
%from a graph $G$ without loops drawn on a surface with selected the main %vertex $M$ and the
%set $\mathcal{A}$ of auxiliary vertices. We assume that there are no %edges between any
%auxiliary vertices. We assume also that for each $A \in \mathcal{A}$ %there exists at least
%one vertex (different from $M$) which is connected with the auxiliary %vertex
%$A$. We assume also that the main vertex $M$ does not have any common %neighbours with any vertex $A\in\mathcal{A}$.
%
%By \emph{the special edges} we will understand the edges connecting $M$ %and $\mathcal{A}$;
%by \emph{the rest of the graph $G'$} we will understand the graph with the
%special edges removed. A \emph{corner} of $G'$ is an inner angle in a %vertex
%between two neighboring edges of $G'$. We denote by $\nast$ the %permutation on the set
%of corners of $G'$ such that each cycle of $\nast$ corresponds to the %corners
%which belong to some face of $G'$, arranged in the 
%cyclic order
%(i.e., by going along the boundary of the face of $G'$, touching it with %the
%right-hand side). 
%For the example on \cref{fig:rysnum} we have $\nast=(1')(1,2,3,4,5,6)$.
%
%
%\begin{figure}
%\sliding{0.95}
%\caption
%{
	%An example of the edge sliding.
%	\protect\subref{fig:rysprzed} A graph with special edges coloured red.
%	\protect\subref{fig:rysw} The graph during the right-handed edge sliding.
%	\protect\subref{fig:ryspo} The graph after the edge sliding.
%	\protect\subref{fig:rysnum} The graph without the special edges with
%	numbered order of the corners.
%}
%\label{fig:sliding}
%\end{figure}
%\todo{\cref{fig:sliding}: czy $\mathcal{A}=\{ A\}$?}
%
%
%Each corner $c$ of a vertex $A \in \mathcal{A}$ can contain some special %edges. 
%The output of the right-handed edge sliding 
%is defined as the graph $G$ 
%in which these special edges from each such a corner $c$ 
%are shifted to the next corner $\nast(c)$
%(see \cref{fig:sliding} for an example where $\mathcal{A}=\{ A\}$
%\todo{ta uwaga powinna się znaleźć w podpisie pod rysunkiem, a nie w %tekście głównym. Moje TODOs nie mogą być umieszczone w podpisach}).
%
%%\todo[inline]{jeśli wierzchołek $M$ jest połączony z wierzchołkiem nie-%auxiliary, efektem ślizgania może być krawędź, która zaczyna się i kończy %w $M$. Czy pętle są ok?}
%
%
%\medskip
%
%Similarly, we define \emph{the left-handed edge sliding} 
%in which the special edges in a corner $c$ are shifted to the corner $%\nast^{-1}(c)$. 
%
%The right-handed edge sliding is the inverse of the left-handed edges
%sliding (for another set of auxiliary vertices, but the same main vertex %and the same special edges).
%%\todo[inline]{to chyba nie jest prawda. Kiedy dokonasz przesunięcia %krawędzi, krawędź przestaje być specjalna, więc przy left-handed %przesuwaniu nie będzie przesuwana, zob.~\cref{fig:sliding}. Co więc Autor %miał na myśli?}
%\todo[inline]{Umiem znaleźć prosty przykład, w którym ten nowy zbiór %auxiliary
%    vertices musi zawierać dwa wierzchołki, które są połączone krawędzią, %a to jest
%    zabronione zgodnie z Twoimi zasadami. Dobrze byłoby a) podać dobrą %definicję
%    ślizgania, i/lub b) jeśli nie potrzebujesz tej odwrotności, skasować %poprzednie
%    zdanie (ale może jednak tej odwrotności będziesz potrzebować, wtedy %warto byłoby
%    pisać zdania, których sens rozumiemy }
%
%In addition, the edge sliding on a graph does not change the
%number of faces of this graph.  

\subsection{The set $X_k$ of maps} 

We consider any map from the set $X_k$ of maps. Any such a
map has one face and an odd number $k$ of edges. We denote the black
vertices by $b_1, b_2$ and the white vertices by $w_1, w_2$. There is
at least one edge between each pair of the vertices of different colours.
Of course, $\degg(b_1)+\degg(b_2)=k$ is an odd number. Without loss of generality 
we may assume that $\degg(b_1)>0$ is odd number and $\degg(b_2)>0$ is even.
Let $k_1, k_2 > 0$ denote the numbers of edges which connect
the vertex $b_1$ with the vertices $w_1, w_2$, respectively. 
Of course $k_1+k_2=\degg(b_1)$ is an odd number. Without loss of the generality 
we may assume that $k_1$ is
even and $k_2$ is odd. The unique map from the
set $X_5$ is shown in \cref{fig:rysodda}.

\subsection{The set $Y_k$ of maps}

We will say that \emph{the vertex $w_j$ is the successor of the vertex
$w_i$} (we denote it by $w_i \rightarrow w_j$) 
if using the cyclic order of the corners we
can move (by walking along the edges and holding them with the right hand)
in two steps from a certain corner $c_i$ of the vertex $w_i$ to a certain
corner $c_j$ of the vertex $w_j$, i.e., $\nast^2(c_i)=c_j$. 

We consider any map from the set $Y_k$. Any such a map has one
face and an odd number $k$ of edges. We denote the black vertex by $b$
and  the white vertices by $w_1, w_2, w_3$. We will write the set $Y_k$ as
a union of three sets which will be defined below.

Let $Y_{k}^{\odd}\subseteq Y_k$ be the set of maps for which there exists an
odd degree white vertex (let us say it is $w_3$) which has the other two white vertices 
as successors, i.e., $w_3\rightarrow w_1$ and $w_3\rightarrow w_2$.
Let $T_k^{\odd}$ be the set of all maps from the set $Y_k^{\odd}$ 
together with the choice of a vertex $w_3$ with this property. 
%\todo{czy chodzi o to, że $T_k^{\odd}$ jest zbiorem par $(\text{mapa},
%\text{wierzchołek})$?}
The unique map from the set $T_{5}^{\odd}$ is shown in \cref{fig:rysoddd}. Clearly
\begin{equation}
\label{ineqodd}
|T_{k}^{\odd}| \geq |Y_{k}^{\odd}|.
\end{equation}

Let $Y_{k}^{\even}\subseteq Y_k$ be the set of maps such that there
exists an even degree vertex (let us say it is $w_3$) which has the other 
two white vertices as successors, i.e., $w_3\rightarrow w_1$ and $w_3\rightarrow w_2$. 
Let $T_k^{\even}$ be the set of all the maps from the set $Y_k^{\even}$ with the choice 
of a vertex $w_3$ with this property.  
The unique map from the set $T_{5}^{\even}$ is shown in \cref{fig:rysevend}.
Clearly
\begin{align}
\label{ineqeven}
|T_{k}^{\even}| \geq |Y_{k}^{\even}|.
\end{align}

Let $Y_{k}^{\rest}\subseteq Y_k$ be the set of maps not included in the
sets $Y_{k}^{\odd}$ and $Y_{k}^{\even}$, i.e. 
\begin{align}
\label{yrestdef}
Y_{k}^{\rest}=Y_k \setminus (Y_{k}^{\odd} \cup Y_{k}^{\even}).
\end{align}

Consider some map $m\in Y_{k}^{\rest}$. 
The set of white vertices with the arrows defined by the successor relation
$\rightarrow$ becomes a directed graph which is strongly connected, and each
white vertex has exactly one outgoing edge to another vertex (and potentially a
loop). Without  loss  of  generality  we  may  assume  that $w_1 \rightarrow w_2$. Thus $w_2 \rightarrow w_3$, $w_3 \rightarrow w_1$.

We claim that $m$ has a vertex of odd degree, greater than $1$. By contradiction,  suppose this is not the case. The map $m$ has at least one odd degree white vertex,
because $\degg(w_1)+\degg(w_2)+\degg(w_3)=k$ is odd. Without loss of generality we may assume that $\degg(w_1)$ is odd. Since $$\degg(w_1)+\degg(w_2)+\degg(w_3)=k>3=1+1+1,$$ %\todo{$>4$ czy $\geq 4$?}
it follows that $\degg(w_1)=1$ and $\degg(w_2),\degg(w_3)$ are even, because $m$
does not have a white vertex with odd degree greater than $1$. The vertex $w_1$
has a unique corner $c_1$.

Naturally $\nast^2(c_1)$ 
is a corner of the vertex $w_2$. 
Note that $\nast^2$ is a permutation of the corners of 
the white vertices which has only one cycle. 
The corners of the white vertices are colored in three colors: $1, 2, 3$
(according to the names of the vertices they are in).  
If a corner $c$ has the color $a$, 
its successor $\nast^2(c)$ has either the color $a$ or $1+a \operatorname{mod} 3$. 
There is only one corner which has the color $1$, so the corner colors 
(arranged in the cyclic order according to the unique cycle of $\nast^2$) are 
$(1, 2, \ldots, 2, 3, \ldots, 3)$.
Since exists only one corner $ c_2 $ of the white vertex $ w_2 $ such 
that $ \nast ^ 2 (c_2) $ is the corner of the vertex $ w_3 $, 
then the clockwise cyclic order of the edges in the black vertex $b$ is as
follows: one edge connected to the vertex $w_1$, a certain number of edges connected
to the vertex $w_2$, a certain number of edges connected to the vertex $w_3$.
It is easy to see that we can replace all edges of the white vertex $w_2$ by a
single edge and get a map with $4$ vertices, the one face and an even number of
edges. We get a contradiction, because such maps does not exist. Therefore, the
map $m$ has white vertex with an odd degree greater than $1$.

Each corner of the vertex $b$ (inner angle in the vertex $b$ between 
two neighboring edges of the map $m$) consists two edges $e_1$ and $e_2=\nast(e_1)$
which connect this vertex with a some vertex/vertices. Let $T_k^{\rest}$
be the set of all the maps from the set $Y_k^{\rest}$ with a selected vertex
denoted by $w_3$ with an odd degree greater than $1$ and his edges decomposition into
two non-empty, disjoint parts $P_1$ and $P_2$ where $P_1$ has an odd
number of edges. Furthermore, we require that each corner 
of the vertex $b$ containing two edges connected to the vertex $w_3$ 
($e_1$ and $e_2$ are connected with the vertex $w_3$)
fulfils one of the properties:
\begin{itemize}
\item $e_1, e_2 \in P_1$,
\item $e_1, e_2 \in P_2$, 
\item $e_1 \in P_1, e_2 \in P_2$. 
\end{itemize}  
An unique example of the map from the set $T_{5}^{\rest}$ is shown in
\cref{fig:rysrestd} ($P_1$ has the blue dashed lines in the picture).
Clearly
\begin{align}
\label{ineqrest}
|T_{k}^{\rest}| \geq |Y_{k}^{\rest}|.
\end{align}

\pagebreak[1] 

\section{Proof of main result}
\subsection{Three bijections}
\paragraph{First bijection}
We consider any map $m$ from the set $X_{k}$. As the set of special edges,
we select all edges connecting the vertex $b_1$ with the white vertices.
Each special edge has the selected end in the white vertex and 
the direction left. We apply the edge sliding on the map $m$.
Then we change the color of the black vertex $b_1$ to white and
its name to $w_3$, and the name of vertex $b_2$ to $b$. Of course, the degree of the vertex $w_3$ does not change and is odd. In
addition, $w_3$ has two distinct successors $w_1$ and $w_2$ different
than $w_3$. We received a map
from the set $T_k^{\odd}$ with a selected vertex $w_3$. Such
transformation is a bijection. \cref{fig:rysodd} shows an example of this
bijection for $k=5$. Thus
\begin{align}
\label{eqodd}
|X_k|=|T_k^{\odd}|.
\end{align}
\vspace{-25pt}
\begin{figure}[H]
\rysodd{0.95}
\caption
{
The example of the first bijection for the $5$ edged map.
	\protect\subref{fig:rysodda} The map from the set $X_{5}$.
	\protect\subref{fig:rysoddb} The map during the edge sliding.
	\protect\subref{fig:rysoddc} The map after the edge sliding.
	\protect\subref{fig:rysoddd} The map from the set $Y_{5}^{\odd}$.
}
\label{fig:rysodd}
\end{figure}
\vspace{-15pt}
\paragraph{Second bijection}
We consider any map $m$ from the set $X_{k}$. As the set of special edges,
we select all edges connecting the vertex $b_2$ with the white vertices.
Each special edge has the selected end in the white vertex and 
the direction left. We apply the edge sliding on the map $m$.
Then we change the color of the black vertex $b_2$ to white and
its name to $w_3$, and the name of vertex $b_1$ to $b$. Of course, the degree of the vertex $w_3$ does not change and is even. In
addition, $w_3$ has two distinct successors $w_1$ and $w_2$ different
than $w_3$. We received a map
from the set $T_k^{\even}$ with a selected vertex $w_3$. Such
transformation is a bijection. \cref{fig:rysodd} shows an example of this
bijection for $k=5$. Thus
\begin{align}
\label{eqeven}
|X_k|=|T_k^{\even}|.
\end{align}
\begin{figure}[H]
\ryseven{0.95}
\caption
{
\protect\subref{fig:rysevena} - \protect\subref{fig:rysevend} 
The example of the second bijection for the $5$ edged map. 
}
\label{fig:ryseven}
\end{figure}

\paragraph{Third bijection}
We consider any map $m$ from the set $X_{k}$. As the set of special edges,
we select all edges connecting the vertex $b_1$ with the white vertices.
Each special edge has the selected end in the white vertex and
the direction left for the edges connected with $w_1$ and the direction right for the edges connected with $w_2$. We apply the edge sliding on the map $m$.
Then we change the color of the black vertex $b_1$ to white and
its name to $w_3$, and the name of vertex $b_2$ to $b$. Of course, the degree of the vertex $w_3$ does not change and is odd. We received a map
from the superset of the set $T_k^{\rest}$ with a selected vertex $w_3$. Such transformation is a bijection. \cref{fig:rysodd} shows an example of this
bijection for $k=5$. Thus
\begin{align}
\label{eqrest}
|X_k| \geq |T_k^{\rest}|.
\end{align}
\vspace{-25pt}
\begin{figure}[H]
\rysrest{0.95}
\caption
{
\protect\subref{fig:rysresta} - \protect\subref{fig:rysrestd} 
The example of the third bijection for the $5$ edged map.
}
\label{fig:rysrest}
\end{figure}

\subsection{End of proof}
Simple calculations give the end the proof
\begin{align*}
3[C_2^2]L_k & =3[R_2^2]K_k-[R_4]K_k \tag{by \ref{formula}}\\ 
& =3|X_k|-|Y_k| \tag{by \ref{xdef},\ref{ydef}}\\ 
& \geq |T_k^{\odd}|+|T_k^{\even}|+|T_k^{\rest}|-|Y_k| \tag{by \ref{eqodd},
\ref{eqeven}, \ref{eqrest}}\\ & \geq |Y_k^{\odd}|+|Y_k^{\even}|+|
Y_k^{\rest}|-|Y_k| \tag{by \ref{ineqodd}, \ref{ineqeven}, \ref{ineqrest}}\\
& =|Y_k^{\odd} \cap Y_k^{\even}| \tag{by \ref{yrestdef}}\\ 
& \geq 0.
\end{align*}
\printbibliography 

\end{document} 
